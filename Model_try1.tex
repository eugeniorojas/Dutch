\documentclass{article}
\usepackage{amsmath}
\usepackage{graphicx}
\usepackage{geometry}
\usepackage{setspace}
\usepackage{epstopdf}
\usepackage{type1cm}
\doublespacing
\addtolength{\parskip}{.4em}
\geometry{papersize={21.59cm,27.94cm}}
\geometry{left=2cm,right=2cm,top=2cm,bottom=2cm}
\title{Natural resources and productive misallocation}
\author{Eugenio Rojas and David Zarruk Valencia}
\date{\today}
\begin{document}

\section{The Model}

We will initially propose a partial equilibrium model composed of one small, open economy (SOE), that has exogenous endowments of labor and a natural resource. The economy has two productive sectors: a competitive intermediate sector that uses labor and the natural resource as inputs to produce a homogeneous and tradeable intermediate good, and a manufacturing sector that produces heterogeneous final goods for domestic consumption, using labor and the intermediate good. The manufacturing sector is made up of a continuum of firms that produce a differentiated final good for consumption, and the variety of final goods expands as new entrants invest on R\&D. That is, there is horizontal growth in the manufacturing sector, driven by an expanding variety of differentiated final goods. \\

The assumption of a small open economy implies that changes in the endowment of the natural resource have no effect on the international price of it. Therefore, we can abstract from general equilibrium considerations, and refer interchangeably to increases on the international price of the natural resource and to increases in the endowment of the resource, since both increase exogenously the revenues to the country. A sudden increase in the endowment of the natural resource or, equivalently, a sudden increase in the international price of the intermediate commodity, have two effects on the economy. First, the higher returns to production of the intermediate good generates a re-allocation of labor from the final good sector to the intermediate sector, reducing investment in innovation and having a long run slowdown of growth. Second, the higher revenues from the resource endowment imply that households have more resources to consume. The impact of both effects on welfare acts in opposite directions, and the overall effect depends on XXXXXX.....

\subsection{Households}

Households choose the amount the consume of each of the final goods, and supply inelastically a unit of labor at each point in time. The preferences of households for the consumption of the differentiated final goods are described by the instantanous utility function:

\[u(t) = \left( \int_0^{N} X_i(t)^{\frac{\epsilon -1}{\epsilon}} di \right)^{\frac{\epsilon}{\epsilon -1}}\]

where $X_i$ is the consumption of final good $i$, $N$ is the mass of final goods produced in the economy, $\epsilon>1$ is the elasticity of substitution across final goods. Household choose their consumption and savings at each point in time, so as to maximize lifetime utility, subject to a budget constraint:

\[\max_{X_i(t)}\int_0^{\infty}e^{-\rho t}log(u(t))dt,	\quad s.t.\]
\[\dot{A}(t) + \int_0^NP_i(t)X_i(t) di= r(t)A(t) + w(t)+\Pi(t) + p(t) \Omega(t)\]


\subsection{Intermediate Sector}

The intermediate sector is composed of a continuum of firms that produce an intermediate good in a competitive market, using labor and the natural resource as inputs. The price of the intermediate good is determined in the international markets, and the assumption of a small, open economy implies that the production decisions of the domestic industry do not affect prices. The intermediate good is produced according to the CES production function:

\[ M = \left( \eta L_M^{\frac{\tau - 1}{\tau}} + (1-\eta)R^{\frac{\tau -1}{\tau}}  \right)^{\frac{\tau}{\tau-1}} \]

Where $R$ is the demand for the natural resource input, $L_M$ is the labor used in the production process, $\eta$ is a weight parameter on the production, and $\tau$ represents the elasticity of substitution between inputs. The factors of production are complements whe $\tau<1$, and substitutes when $\tau>1$.

\subsection{Final goods sector}

Firms operating in the final goods sector produce accoding to the production function:

\[  \]

\end{document}











